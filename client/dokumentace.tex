\documentclass[12pt,a4paper]{article}
\usepackage[utf8]{inputenc}
\usepackage[czech]{babel}
\usepackage{graphicx}
\usepackage{listings}
\usepackage{xcolor}
\usepackage{hyperref}
\usepackage{geometry}
\usepackage{enumitem}
\usepackage{fancyhdr}

\geometry{margin=2.5cm}

% Nastavení pro kód
\definecolor{codegreen}{rgb}{0,0.6,0}
\definecolor{codegray}{rgb}{0.5,0.5,0.5}
\definecolor{codepurple}{rgb}{0.58,0,0.82}
\definecolor{backcolour}{rgb}{0.95,0.95,0.92}

\lstdefinestyle{pythonstyle}{
    backgroundcolor=\color{backcolour},   
    commentstyle=\color{codegreen},
    keywordstyle=\color{magenta},
    numberstyle=\tiny\color{codegray},
    stringstyle=\color{codepurple},
    basicstyle=\ttfamily\footnotesize,
    breakatwhitespace=false,         
    breaklines=true,                 
    captionpos=b,                    
    keepspaces=true,                 
    numbers=left,                    
    numbersep=5pt,                  
    showspaces=false,                
    showstringspaces=false,
    showtabs=false,                  
    tabsize=2,
    language=Python
}

\lstset{style=pythonstyle}

% Záhlaví a zápatí
\pagestyle{fancy}
\fancyhf{}
\rhead{Klientská aplikace pro hru Mariáš}
\lhead{Technická dokumentace}
\rfoot{Strana \thepage}

\title{\textbf{Technická dokumentace} \\ Klientská aplikace pro karetní hru Mariáš}
\author{Projekt UPS}
\date{\today}

\begin{document}

\maketitle
\newpage

\tableofcontents
\newpage

\section{Úvod}

Tento dokument popisuje architekturu a implementaci klientské aplikace pro online karetní hru Mariáš. Aplikace je napsána v jazyce Python s využitím knihovny Pygame pro grafické rozhraní a implementuje binární síťový protokol pro komunikaci se serverem.

\subsection{Účel aplikace}

Klientská aplikace slouží jako uživatelské rozhraní pro hru Mariáš, která umožňuje:
\begin{itemize}
    \item Připojení k hernímu serveru přes TCP/IP
    \item Grafické zobrazení herního stavu
    \item Interakci s hráčem prostřednictvím GUI
    \item Automatické znovupřipojení při výpadku spojení
    \item Zobrazení pravidel hry
\end{itemize}

\subsection{Technologie}

Projekt využívá následující technologie:
\begin{itemize}
    \item \textbf{Python 3.x} -- programovací jazyk
    \item \textbf{Pygame} -- knihovna pro grafické rozhraní a herní logiku
    \item \textbf{Socket} -- síťová komunikace
    \item \textbf{Threading} -- asynchronní zpracování síťových zpráv
    \item \textbf{Enum} -- typově bezpečné výčtové typy
\end{itemize}

\section{Architektura aplikace}

Aplikace je navržena podle principů objektově orientovaného programování a je rozdělena do několika logických celků.

\subsection{Struktura projektu}

\begin{verbatim}
client/
├── main.py                    # Vstupní bod aplikace
├── images/                    # Grafické assety (karty, pozadí)
└── src/
    ├── Gui.py                # Hlavní třída GUI a stavový automat
    ├── GameManager.py        # Správa herní logiky a vykreslování
    ├── Client/
    │   ├── ClientManager.py  # Síťová komunikace
    │   └── Protocol.py       # Definice protokolu
    ├── Game/
    │   ├── Game.py          # Herní stav a pravidla
    │   ├── Card.py          # Reprezentace karet
    │   └── Player.py        # Reprezentace hráčů
    └── View/
        ├── GuiManager.py    # Vykreslování UI komponent
        ├── Obstacles.py     # UI prvky (tlačítka, inputy)
        └── Validator.py     # Validace uživatelských vstupů
\end{verbatim}

\subsection{Diagram komponent}

Aplikace se skládá z následujících hlavních komponent:

\begin{itemize}
    \item \textbf{Gui} -- Řídící třída, která spravuje stavový automat aplikace
    \item \textbf{ClientManager} -- Zajišťuje síťovou komunikaci se serverem
    \item \textbf{GameManager} -- Spravuje herní logiku a vykreslování hry
    \item \textbf{GuiManager} -- Vykresluje UI komponenty (lobby, čekání, hra)
    \item \textbf{Game} -- Obsahuje herní stav a pravidla Mariáše
\end{itemize}

\section{Hlavní komponenty}

\subsection{Gui -- Hlavní třída aplikace}

Třída \texttt{Gui} je centrálním bodem aplikace a implementuje stavový automat s následujícími stavy:

\begin{description}
    \item[LOBBY] -- Úvodní obrazovka pro zadání IP, portu a přezdívky
    \item[CONNECTING] -- Probíhá připojování k serveru
    \item[WAITING] -- Čekání na ostatní hráče
    \item[PLAYING] -- Probíhá hra
    \item[RECONNECTING] -- Probíhá znovupřipojení po výpadku
    \item[DISCONNECTION] -- Odpojení od serveru
    \item[HELP] -- Zobrazení nápovědy a pravidel
\end{description}

\subsubsection{Klíčové metody}

\begin{itemize}
    \item \texttt{run()} -- Hlavní herní smyčka s optimalizací vykreslování
    \item \texttt{setup\_client\_callbacks()} -- Nastavení callbacků pro zprávy od serveru
    \item \texttt{handle\_server\_message()} -- Zpracování příchozích zpráv
    \item \texttt{connect\_to\_server()} -- Připojení k serveru s validací vstupů
\end{itemize}

\subsection{ClientManager -- Síťová komunikace}

Třída \texttt{ClientManager} zajišťuje veškerou síťovou komunikaci s herním serverem.

\subsubsection{Funkce}

\begin{itemize}
    \item Připojení k serveru přes TCP socket
    \item Asynchronní příjem zpráv v samostatném vlákně
    \item Automatické znovupřipojení při výpadku spojení
    \item Heartbeat mechanismus pro detekci výpadků
    \item Fronta zpráv pro thread-safe zpracování
\end{itemize}

\subsubsection{Reconnect mechanismus}

Při ztrátě spojení se klient automaticky pokouší o znovupřipojení:
\begin{enumerate}
    \item Detekce výpadku (timeout při čtení nebo chyba socketu)
    \item Spuštění reconnect vlákna
    \item Exponenciální backoff (1s, 2s, 4s, 8s, 16s)
    \item Maximálně 5 pokusů o znovupřipojení
    \item Odeslání RECONNECT\_REQ zprávy po úspěšném připojení
\end{enumerate}

\subsection{Protocol -- Síťový protokol}

Aplikace používá binární protokol pro komunikaci se serverem.

\subsubsection{Formát zprávy}

\begin{verbatim}
[ 2B Velikost | 1B Packet ID | 1B Client Number | 1B Type | Data ]
\end{verbatim}

\begin{itemize}
    \item \textbf{Velikost} (2B) -- Celková velikost zprávy včetně hlavičky
    \item \textbf{Packet ID} (1B) -- Identifikátor paketu
    \item \textbf{Client Number} (1B) -- Číslo klienta
    \item \textbf{Type} (1B) -- Typ zprávy (viz MessageType)
    \item \textbf{Data} -- Pole oddělená znakem '|'
\end{itemize}

\subsubsection{Typy zpráv}

\textbf{Server → Client:}
\begin{itemize}
    \item ERROR (0) -- Chybová zpráva
    \item WELCOME (2) -- Přivítání po připojení
    \item STATE (3) -- Aktualizace herního stavu
    \item GAME\_START (4) -- Začátek hry
    \item RESULT (5) -- Výsledek hry
    \item YOUR\_TURN (8) -- Informace o tahu hráče
    \item WAIT\_LOBBY (9) -- Čekání v lobby
\end{itemize}

\textbf{Client → Server:}
\begin{itemize}
    \item CONNECT (100) -- Požadavek na připojení
    \item RECONNECT\_REQ (101) -- Požadavek na znovupřipojení
    \item READY\_REQ (102) -- Oznámení připravenosti
    \item CARD (103) -- Zahraná karta
    \item TRICK (104) -- Prázdná zpráva pro štych
    \item BIDDING (105) -- Licitace
    \item HEARTBEAT (107) -- Heartbeat zpráva
\end{itemize}

\subsection{GameManager -- Správa hry}

Třída \texttt{GameManager} spojuje herní logiku s grafickým rozhraním.

\subsubsection{Odpovědnosti}

\begin{itemize}
    \item Vykreslování karet hráče
    \item Zobrazení zahraných karet na stole
    \item Vykreslování licitačních tlačítek
    \item Zpracování kliknutí na karty
    \item Parsování herních dat ze serveru
    \item Zobrazení výsledků hry
\end{itemize}

\subsubsection{Parsování dat}

GameManager obsahuje metody pro převod dat ze serveru do objektů:
\begin{itemize}
    \item \texttt{card\_reader()} -- Převod stringu na objekt Card
    \item \texttt{player\_reader()} -- Vytvoření objektu Player z dat
    \item \texttt{state\_reader()} -- Aktualizace herního stavu
    \item \texttt{game\_start\_reader()} -- Inicializace nové hry
\end{itemize}

\subsection{Game -- Herní logika}

Třída \texttt{Game} reprezentuje stav hry Mariáš.

\subsubsection{Herní stavy}

\begin{description}
    \item[LICITACE\_TRUMF] -- Licitace trumfové barvy
    \item[LICITACE\_TALON] -- Licitace talonu
    \item[LICITACE\_HRA] -- Licitace typu hry
    \item[LICITACE\_DOBRY\_SPATNY] -- Licitace dobrý/špatný
    \item[LICITACE\_BETL\_DURCH] -- Licitace betl/durch
    \item[HRA] -- Probíhá hra
    \item[END] -- Konec hry
\end{description}

\subsubsection{Atributy}

\begin{itemize}
    \item \texttt{players} -- Seznam hráčů
    \item \texttt{state} -- Aktuální stav hry
    \item \texttt{played\_cards} -- Karty zahrané ve štýchu
    \item \texttt{active\_player} -- Zda je tento klient na tahu
    \item \texttt{licitator} -- Hráč, který vyhrál licitaci
    \item \texttt{trumph} -- Trumfová barva
    \item \texttt{mode} -- Herní mód (HRA, BETL, DURCH)
\end{itemize}

\subsection{Card -- Reprezentace karty}

Třída \texttt{Card} reprezentuje jednu hrací kartu.

\subsubsection{Vlastnosti karty}

\begin{itemize}
    \item \textbf{Hodnota (rank)} -- A, K, Q, J, X (10), IX (9), VIII (8), VII (7)
    \item \textbf{Barva (suit)} -- Srdce (♥), Kule (♦), Žaludy (♣), Listy (♠)
\end{itemize}

\subsubsection{Metody}

\begin{itemize}
    \item \texttt{get\_image()} -- Vrátí pygame Surface s obrázkem karty
    \item \texttt{get\_value(mode)} -- Vrátí hodnotu karty podle herního módu
    \item \texttt{\_\_eq\_\_()} -- Porovnání dvou karet
\end{itemize}

\subsection{Player -- Reprezentace hráče}

Třída \texttt{Player} reprezentuje jednoho hráče ve hře.

\subsubsection{Atributy}

\begin{itemize}
    \item \texttt{number} -- Číslo hráče (0-2)
    \item \texttt{nickname} -- Přezdívka hráče
    \item \texttt{hand} -- Ruka hráče (objekt Hand)
    \item \texttt{status} -- Stav hráče
\end{itemize}

\subsubsection{Třída Hand}

Třída \texttt{Hand} spravuje karty v ruce hráče:
\begin{itemize}
    \item \texttt{sort(mode)} -- Seřadí karty podle barvy a hodnoty
    \item \texttt{find\_card\_in\_hand()} -- Najde kartu v ruce
    \item \texttt{add\_card()} -- Přidá kartu do ruky
    \item \texttt{remove\_card()} -- Odebere kartu z ruky
\end{itemize}

\subsection{GuiManager -- Vykreslování UI}

Třída \texttt{GuiManager} zajišťuje vykreslování všech UI komponent.

\subsubsection{Obrazovky}

\begin{itemize}
    \item \texttt{draw\_lobby()} -- Lobby s input fieldy
    \item \texttt{draw\_connecting()} -- Obrazovka připojování
    \item \texttt{draw\_waiting()} -- Čekání na hráče s progress barem
    \item \texttt{draw\_reconnecting()} -- Reconnecting s animací
    \item \texttt{draw\_help()} -- Nápověda s pravidly hry
\end{itemize}

\subsubsection{UI komponenty}

GuiManager používá následující komponenty z modulu \texttt{Obstacles}:
\begin{itemize}
    \item \texttt{InputBox} -- Textové vstupní pole
    \item \texttt{Button} -- Klikací tlačítko
    \item \texttt{HelpButton} -- Tlačítko nápovědy
\end{itemize}

\section{Tok dat a komunikace}

\subsection{Připojení k serveru}

\begin{enumerate}
    \item Uživatel zadá IP, port a přezdívku v lobby
    \item Validace vstupů (InputValidator)
    \item Gui volá \texttt{connect\_to\_server()}
    \item ClientManager vytvoří TCP socket a připojí se
    \item Odeslání CONNECT zprávy s přezdívkou
    \item Server odpoví WELCOME zprávou s číslem klienta
    \item Přechod do stavu WAITING
\end{enumerate}

\subsection{Herní smyčka}

\begin{enumerate}
    \item Server odešle GAME\_START zprávu
    \item GameManager inicializuje hru a hráče
    \item Přechod do stavu PLAYING
    \item Server průběžně posílá STATE zprávy s aktualizacemi
    \item Při tahu hráče server pošle YOUR\_TURN
    \item Hráč klikne na kartu
    \item Odeslání CARD zprávy serveru
    \item Server validuje tah a pošle nový STATE
    \item Opakování dokud hra neskončí
    \item Server pošle RESULT zprávu s výsledkem
\end{enumerate}

\subsection{Zpracování zpráv}

Všechny zprávy od serveru procházejí následujícím tokem:

\begin{enumerate}
    \item ClientManager přijme data v \texttt{\_listen\_loop()}
    \item Zpráva je vložena do thread-safe fronty
    \item \texttt{\_process\_message\_queue()} zpracuje zprávu v hlavním vlákně
    \item Volání příslušného callbacku v Gui
    \item Aktualizace stavu a označení pro překreslení
    \item Vykreslení změn v hlavní smyčce
\end{enumerate}

\section{Optimalizace a výkon}

\subsection{Vykreslování}

Aplikace používá optimalizované vykreslování:
\begin{itemize}
    \item Cachování pozadí (vytvoří se pouze jednou)
    \item Dirty flag systém -- překreslení pouze při změně
    \item Omezení FPS na 60 pro snížení zátěže CPU
    \item Vykreslování pouze viditelných prvků
\end{itemize}

\subsection{Síťová komunikace}

\begin{itemize}
    \item Asynchronní příjem v samostatném vlákně
    \item Fronta zpráv pro thread-safe zpracování
    \item Heartbeat každých 10 sekund
    \item Timeout 30 sekund pro detekci výpadku
    \item Exponenciální backoff při reconnectu
\end{itemize}

\section{Validace vstupů}

Třída \texttt{InputValidator} zajišťuje validaci uživatelských vstupů:

\begin{itemize}
    \item \textbf{IP adresa} -- Kontrola formátu IPv4 (regex)
    \item \textbf{Port} -- Číslo v rozsahu 1-65535
    \item \textbf{Přezdívka} -- Délka 1-20 znaků, alfanumerické znaky
\end{itemize}

\section{Zpracování chyb}

\subsection{Síťové chyby}

\begin{itemize}
    \item Timeout při připojení -- zobrazení chyby v GUI
    \item Ztráta spojení -- automatický reconnect
    \item Chyba při parsování zprávy -- logování a pokračování
    \item Neplatná zpráva od serveru -- zobrazení ERROR
\end{itemize}

\subsection{Herní chyby}

\begin{itemize}
    \item Neplatný tah -- server odpoví INVALID zprávou
    \item Zobrazení chybové hlášky v GUI
    \item Hráč může zkusit jiný tah
\end{itemize}

\section{Rozšiřitelnost}

Aplikace je navržena s ohledem na budoucí rozšíření:

\begin{itemize}
    \item Modulární architektura -- snadné přidání nových funkcí
    \item Oddělení logiky od prezentace
    \item Protokol podporuje rozšíření o nové typy zpráv
    \item GUI komponenty jsou znovupoužitelné
\end{itemize}

\subsection{Možná rozšíření}

\begin{itemize}
    \item Podpora více herních místností
    \item Chat mezi hráči
    \item Statistiky a historie her
    \item Různé grafické motivy
    \item Zvukové efekty
    \item Podpora pro více jazyků
\end{itemize}

\section{Závěr}

Klientská aplikace pro hru Mariáš je robustní a dobře strukturovaná aplikace, která poskytuje uživatelsky přívětivé rozhraní pro online hraní této tradiční karetní hry. Implementace síťového protokolu, automatického znovupřipojení a optimalizovaného vykreslování zajišťuje plynulý herní zážitek i při nestabilním připojení.

Modulární architektura a čistý kód umožňují snadnou údržbu a budoucí rozšíření aplikace o nové funkce.

\end{document}
